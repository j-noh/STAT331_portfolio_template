% Options for packages loaded elsewhere
\PassOptionsToPackage{unicode}{hyperref}
\PassOptionsToPackage{hyphens}{url}
\PassOptionsToPackage{dvipsnames,svgnames,x11names}{xcolor}
%
\documentclass[
  letterpaper,
  DIV=11,
  numbers=noendperiod]{scrartcl}

\usepackage{amsmath,amssymb}
\usepackage{lmodern}
\usepackage{iftex}
\ifPDFTeX
  \usepackage[T1]{fontenc}
  \usepackage[utf8]{inputenc}
  \usepackage{textcomp} % provide euro and other symbols
\else % if luatex or xetex
  \usepackage{unicode-math}
  \defaultfontfeatures{Scale=MatchLowercase}
  \defaultfontfeatures[\rmfamily]{Ligatures=TeX,Scale=1}
\fi
% Use upquote if available, for straight quotes in verbatim environments
\IfFileExists{upquote.sty}{\usepackage{upquote}}{}
\IfFileExists{microtype.sty}{% use microtype if available
  \usepackage[]{microtype}
  \UseMicrotypeSet[protrusion]{basicmath} % disable protrusion for tt fonts
}{}
\makeatletter
\@ifundefined{KOMAClassName}{% if non-KOMA class
  \IfFileExists{parskip.sty}{%
    \usepackage{parskip}
  }{% else
    \setlength{\parindent}{0pt}
    \setlength{\parskip}{6pt plus 2pt minus 1pt}}
}{% if KOMA class
  \KOMAoptions{parskip=half}}
\makeatother
\usepackage{xcolor}
\setlength{\emergencystretch}{3em} % prevent overfull lines
\setcounter{secnumdepth}{-\maxdimen} % remove section numbering
% Make \paragraph and \subparagraph free-standing
\ifx\paragraph\undefined\else
  \let\oldparagraph\paragraph
  \renewcommand{\paragraph}[1]{\oldparagraph{#1}\mbox{}}
\fi
\ifx\subparagraph\undefined\else
  \let\oldsubparagraph\subparagraph
  \renewcommand{\subparagraph}[1]{\oldsubparagraph{#1}\mbox{}}
\fi


\providecommand{\tightlist}{%
  \setlength{\itemsep}{0pt}\setlength{\parskip}{0pt}}\usepackage{longtable,booktabs,array}
\usepackage{calc} % for calculating minipage widths
% Correct order of tables after \paragraph or \subparagraph
\usepackage{etoolbox}
\makeatletter
\patchcmd\longtable{\par}{\if@noskipsec\mbox{}\fi\par}{}{}
\makeatother
% Allow footnotes in longtable head/foot
\IfFileExists{footnotehyper.sty}{\usepackage{footnotehyper}}{\usepackage{footnote}}
\makesavenoteenv{longtable}
\usepackage{graphicx}
\makeatletter
\def\maxwidth{\ifdim\Gin@nat@width>\linewidth\linewidth\else\Gin@nat@width\fi}
\def\maxheight{\ifdim\Gin@nat@height>\textheight\textheight\else\Gin@nat@height\fi}
\makeatother
% Scale images if necessary, so that they will not overflow the page
% margins by default, and it is still possible to overwrite the defaults
% using explicit options in \includegraphics[width, height, ...]{}
\setkeys{Gin}{width=\maxwidth,height=\maxheight,keepaspectratio}
% Set default figure placement to htbp
\makeatletter
\def\fps@figure{htbp}
\makeatother

\KOMAoption{captions}{tableheading}
\makeatletter
\makeatother
\makeatletter
\makeatother
\makeatletter
\@ifpackageloaded{caption}{}{\usepackage{caption}}
\AtBeginDocument{%
\ifdefined\contentsname
  \renewcommand*\contentsname{Table of contents}
\else
  \newcommand\contentsname{Table of contents}
\fi
\ifdefined\listfigurename
  \renewcommand*\listfigurename{List of Figures}
\else
  \newcommand\listfigurename{List of Figures}
\fi
\ifdefined\listtablename
  \renewcommand*\listtablename{List of Tables}
\else
  \newcommand\listtablename{List of Tables}
\fi
\ifdefined\figurename
  \renewcommand*\figurename{Figure}
\else
  \newcommand\figurename{Figure}
\fi
\ifdefined\tablename
  \renewcommand*\tablename{Table}
\else
  \newcommand\tablename{Table}
\fi
}
\@ifpackageloaded{float}{}{\usepackage{float}}
\floatstyle{ruled}
\@ifundefined{c@chapter}{\newfloat{codelisting}{h}{lop}}{\newfloat{codelisting}{h}{lop}[chapter]}
\floatname{codelisting}{Listing}
\newcommand*\listoflistings{\listof{codelisting}{List of Listings}}
\makeatother
\makeatletter
\@ifpackageloaded{caption}{}{\usepackage{caption}}
\@ifpackageloaded{subcaption}{}{\usepackage{subcaption}}
\makeatother
\makeatletter
\@ifpackageloaded{tcolorbox}{}{\usepackage[many]{tcolorbox}}
\makeatother
\makeatletter
\@ifundefined{shadecolor}{\definecolor{shadecolor}{rgb}{.97, .97, .97}}
\makeatother
\makeatletter
\makeatother
\ifLuaTeX
  \usepackage{selnolig}  % disable illegal ligatures
\fi
\IfFileExists{bookmark.sty}{\usepackage{bookmark}}{\usepackage{hyperref}}
\IfFileExists{xurl.sty}{\usepackage{xurl}}{} % add URL line breaks if available
\urlstyle{same} % disable monospaced font for URLs
\hypersetup{
  pdftitle={Final Grade Reflection},
  pdfauthor={Jun Noh},
  colorlinks=true,
  linkcolor={blue},
  filecolor={Maroon},
  citecolor={Blue},
  urlcolor={Blue},
  pdfcreator={LaTeX via pandoc}}

\title{Final Grade Reflection}
\author{Jun Noh}
\date{}

\begin{document}
\maketitle
\ifdefined\Shaded\renewenvironment{Shaded}{\begin{tcolorbox}[frame hidden, borderline west={3pt}{0pt}{shadecolor}, interior hidden, breakable, enhanced, sharp corners, boxrule=0pt]}{\end{tcolorbox}}\fi

In this document, you make a data-based argument for the grade you've
earned in this course. Your argument should include evidence from the
supporting artifacts you've provided.

\hypertarget{the-output-document-should-be-a-pdf-or-a-word-document-as-it-should-be-a-maximum-of-2-pages.}{%
\subsection{\texorpdfstring{The output document should be a PDF or a
Word Document, as it should be a \textbf{maximum} of
2-pages.}{The output document should be a PDF or a Word Document, as it should be a maximum of 2-pages.}}\label{the-output-document-should-be-a-pdf-or-a-word-document-as-it-should-be-a-maximum-of-2-pages.}}

I believe I have earned an A in this class because I have been able to
meet every learning target, challenge myself to make better
documents/code, and have contributed to being a good classmate. In lab 4
\#1 I met targets WD-1 and R-1. Target WD-6 was shown in lab 4 \#9 and
WD-4, WD-7, PE-4 in lab 4 \#10. WD-3, PE-4 was shown in lab 4 \# 4.
DVS-2 was in lab 5 \#3 along with DVS-1 which was shown in lab 5 \#4, 5,
6. DSM-2 was found in lab 9 \#2, 3, 4. WD-5 was located in preview
activity 4 \#4-6. DVS-4 was in lab 9 \#1 along with PE-1 which was found
in lab 9 \#1a. DVS-6 and DVS-7 can both be found in challenge 9 \#5.
DVS-5 was found in lab 7 \#2 along with R-3, PE-2 in lab 7 \#4 and \#8.
R-3 can also be found in lab 3 \#10. Lab 3 \#9 contains WD-2. PE-3 was
found in Lab 8 \#2 and 6. DSM-1 is in practice activity 9 \#4 and 4a.
DVS-3 was in challenge 4 \#3 as well as lab 4 \#11, and lab 5 \#6. I
think throughout these examples I have shown learning target R-2 by
using tidy code and making my labs/challenges reproducible. Other than
learning targets I have achieved throughout the class, I have also shown
commitment to continued learning. Throughout the challenges I have tried
to add unique elements such as challenge 2 option 3 where I used
annotations or challenge 4 where I scaled my graph. Additionally, I have
consistently been doing revisions, looking through discord, asking my
team members for help, or going to office hours to get clarification on
certain topics that I might be confused on. I think I have grown more
comfortable working as a team member throughout the quarter, I am able
to share my ideas and also talk through others' suggestions. I am not
hesitant to reach out to my group members for help, and I also try to
share things I have found useful that might also help them. Furthermore,
I try to come to class with my preview activities completed, so that I
am not behind during the practice activity. Along with the fact that I
complete preview activities before class and show up to every class I
have also answered a few questions on discord. I think these two factors
have shown that I contribute to a good classroom environment. Finally, I
think I have met my personal goals that I had set earlier in the year. I
am more comfortable working with R and using the tools within R to
accomplish various tasks, I also find working in a group to be easier
now which is another goal I had initially set at the start of the year.



\end{document}
